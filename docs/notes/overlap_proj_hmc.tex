\documentstyle{report}

% Somewhat wider and taller page than in art12.sty
\topmargin -0.4in  \headsep 0.4in  \textheight 9.0in
\oddsidemargin 0.25in  \evensidemargin 0.25in  \textwidth 6in

\footnotesep 14pt
\floatsep 28pt plus 2pt minus 4pt      % Nominal is double what is in art12.sty
\textfloatsep 40pt plus 2pt minus 4pt
\intextsep 28pt plus 4pt minus 4pt

\newcommand{\be}{\begin{equation}}
\newcommand{\ee}{\end{equation}}

\begin{document}

\centerline{ Inclusion of projection operators in the fermionic force}
\rightline{Rajamani Narayanan}
\rightline{8/6/99}

In this note, I will show how to include projection operators
in the fermionic force. Our starting point is the eigenvalue
equation for the hermitian Wilson-Dirac operator:
\be
H_w\psi_n = \lambda_h \psi_n\label{eq:eigen}
\ee
We define the following projection operators:
\be
P_n = \psi_n\psi^\dagger_n; \ \ \ \ Q_n=1-P_n;\ \ \ \  Q=1-\sum_nP_n
\ee
We will not assume any specific form for $\epsilon(H_w)$ in this note.
What follows applies for any realization of this matrix including
a five-dimensional realization that includes extra fields.
I assume that the variation is taken with respect to a single
scalar variable.
We start by writing
\be
\epsilon = Q\epsilon Q + \sum_n \epsilon(\lambda_n) P_n
\ee
Then we have,
\be
{\delta \epsilon\over \delta U} =
Q {\delta \epsilon\over \delta U} Q - \sum_n\Bigl[
{\delta P_n\over \delta U} \epsilon Q + Q\epsilon {\delta P_n\over \delta U}
-\epsilon(\lambda_n) {\delta P_n\over \delta U} \Bigr]
\ee
We will not concern ourselves with the first term in the last equation.
This has been dealt with in a previous note. We have assumed that all
$\lambda_n\ne 0$ in the differentiation above.
The variation of $P_n$ is given by
\be
{\delta P_n\over \delta U} = {\delta \psi_n\over \delta U}\psi^\dagger_n
+ \psi_n {\delta \psi^\dagger_n\over \delta U}\label{eq:deriv_p}
\ee
Differentiation of (\ref{eq:eigen}) with respect to $U$ gives,
\be{\delta \lambda_n\over \delta U} =
\psi_n^\dagger { \delta H_w\over \delta U} \psi_n
\ee
and
\be
(H_w-\lambda_n) {\delta \psi_n\over \delta U} =
\Bigl[ \psi_n^\dagger { \delta H_w\over \delta U} \psi_n
- { \delta H_w\over \delta U} \Bigr] \psi_n\label{deriv_vec}
\ee
Note that the right hand side of equation (\ref{deriv_vec}) is orthogonal to $\psi_n$.
We can assume that $P_n {\delta \psi_n\over \delta U} =0$ and write
\be
{\delta \psi_n\over \delta U} =
Q_n {1\over H_w -\lambda_n} Q_n \Bigl[ \psi_n^\dagger 
{ \delta H_w\over \delta U} \psi_n
- { \delta H_w\over \delta U} \Bigr] \psi_n
\label{eq:deriv_psi}
\ee
Note that the presence of the $Q_n$ factors on the right hand side
simply ensures the orthogonality assumed above.
Using (\ref{eq:deriv_psi}) in (\ref{eq:deriv_p}), we get
\be
{\delta P_n\over \delta U} =
Q_n {1\over H_w -\lambda_n} Q_n \Bigl[ \psi_n^\dagger 
{ \delta H_w\over \delta U} \psi_n
- { \delta H_w\over \delta U} \Bigr] P_n 
+ P_n \Bigl[ \psi_n^\dagger 
{ \delta H_w\over \delta U} \psi_n
- { \delta H_w\over \delta U} \Bigr] Q_n {1\over H_w -\lambda_n} Q_n
\ee
Since
\be
P_n\epsilon = \epsilon(\lambda_n) P_n;\ \ \ \ \ P_n Q =0
\ee
we have
\be
{\delta P_n\over \delta U} \epsilon Q =
P_n \Bigl[ \psi_n^\dagger 
{ \delta H_w\over \delta U} \psi_n
- { \delta H_w\over \delta U} \Bigr] Q_n {1\over H_w -\lambda_n} \epsilon Q
\ee
and
\be
Q\epsilon {\delta P_n\over \delta U} =
\epsilon Q_n {1\over H_w -\lambda_n} Q_n \Bigl[ \psi_n^\dagger 
{ \delta H_w\over \delta U} \psi_n
- { \delta H_w\over \delta U} \Bigr] P_n 
\ee

\end{document}

\documentclass[12pt]{article}
\usepackage{amsmath}

% Somewhat wider and taller page than in art12.sty
%\topmargin -0.4in  \headsep 0.0in  \textheight 9.0in
%\oddsidemargin 0.25in  \evensidemargin 0.25in  \textwidth 6.5in

\footnotesep 14pt
\floatsep 28pt plus 2pt minus 4pt      % Nominal is double what is in art12.sty
\textfloatsep 40pt plus 2pt minus 4pt
\intextsep 28pt plus 4pt minus 4pt

\topmargin -1cm
\headsep 0mm
\oddsidemargin 1mm
\evensidemargin 1mm
\textwidth 170mm
\textheight 24cm

\begin{document}
\begin{center}
\Large Correlation functions and Sequential sources for 
$N\rightarrow N$, $\Delta\rightarrow N$ \\
and the $\Delta\rightarrow\Delta$ 
form factors
\vspace{2mm}

21 July 2005
\end{center}

\section{Useful definitions}

We will consider the nucleon and $\Delta^+$ interpolating fields
(including Dirac index $\alpha$ and Lorentz index $\sigma$), such as
\begin{eqnarray*}
\chi^N_{\alpha L}(x) &=&
    \epsilon^{abc}\left(d^{Ta}(x)C\gamma_5u^b(x)\right)u_\alpha^c(x) \\
\chi^{\Delta^+}_{\sigma,\alpha L}(x) &=&
    \epsilon^{abc}\left[2\left(d^{Ta}(x)C\gamma_\sigma u^b(x)\right)u_\alpha^c(x)
                        +\left(u^{Ta}(x)C\gamma_\sigma u^b(x)\right)d_\alpha^c(x)\right]
\end{eqnarray*}
or some smeared version of these, denoted $\chi^N_{\alpha S}(x)$ or
$\chi^{\Delta^+}_{\sigma,\alpha S}(x)$.  In the following, the octet
and decuplet baryons will be labelled by $N$ and $\Delta$,
respectively.  However, the results may be applied to any of the octet
to decuplet transitions.

We will define some functions that are convenient for writing the correlation
functions.
%
\begin{itemize}
\item
\verb|quarkContract13(a,b)|\\
Epsilon contract 2 quark propagators and return a quark propagator.
This is used for diquark constructions. Eventually, it could handle larger
Nc. 
The numbers represent which spin index to sum over.
   
The sources and targets must all be propagators but not
necessarily of the same lattice type. Effectively, one can use
this to construct an anti-quark from a di-quark contraction. In
explicit index form, the operation  \verb|quarkContract13| does

\[
target^{k' k}_{\alpha\beta} =
 \epsilon^{i j k}\epsilon^{i' j' k'}* source1^{i i'}_{\rho\alpha}* source2^{j j'}_{\rho\beta}
\]

\item
\verb|quarkContract14(a,b)|\\
Epsilon contract 2 quark propagators and return a quark propagator.
   
\[
target^{k' k}_{\alpha\beta} =
    \epsilon^{i j k}\epsilon^{i' j' k'}*source1^{i i'}_{\rho\alpha}*source2^{j j'}_{\beta\rho}
\]

\item
\verb|quarkContract23(a,b)|\\
Epsilon contract 2 quark propagators and return a quark propagator.
   
\[
target^{k' k}_{\alpha\beta} =
    \epsilon^{i j k}\epsilon^{i' j' k'}*source1^{i i'}_{\alpha\rho}*source2^{j j'}_{\rho\beta}\]

\item
\verb|quarkContract24(a,b)|\\
Epsilon contract 2 quark propagators and return a quark propagator.
   
\[
target^{k' k}_{\alpha\beta} =
    \epsilon^{i j k}\epsilon^{i' j' k'}*source1^{i i'}_{\rho\alpha}*source2^{j j'}_{\beta\rho}
\]

\item
\verb|quarkContract12(a,b)|\\
Epsilon contract 2 quark propagators and return a quark propagator.

\[target^{k' k}_{\alpha\beta} =
    \epsilon^{i j k}\epsilon^{i' j' k'}*source1^{i i'}_{\rho\rho}*source2^{j j'}_{\alpha\beta}
\]

\item
\verb|quarkContract34(a,b)|\\
Epsilon contract 2 quark propagators and return a quark propagator.
\[target^{k' k}_{\alpha\beta} =
    \epsilon^{i j k}\epsilon^{i' j' k'}*source1^{i i'}_{\alpha\beta}*source2^{j j'}_{\rho\rho}\]

\item
\verb|colorContract(a,b,c)|\\
Epsilon contract 3 color primitives and return a primitive scalar.
The sources and targets must all be of the same primitive type (a matrix or vector)
but not necessarily of the same lattice type. In
explicit index form, the operation  colorContract does
\[
target =
  \epsilon^{i j k}\epsilon^{i' j' k'}* source1^{i i'}* source2^{j j'}*source3^{k k'}
\]
or
\[
target =
 \epsilon^{i j k}* source1^{i}* source2^{j}*source3^{k}
\]

\end{itemize}

We will define correlation functions for interpolating fields generically
labelled as $B$. For example, for $N\rightarrow N$ we want
\begin{gather*}
\begin{split}
\langle B^{123}_\gamma(x) \bar{B}^{123}_{\bar\gamma}(0)\rangle
 &= -\varepsilon^{abc}\varepsilon^{\bar{a}\bar{b}\bar{c}} \langle
  \left(\psi^{T a}_{1,\alpha}(C\gamma_5)_{\alpha\beta}\psi^b_{2,\beta}\right)
        \psi^c_{3,\gamma}(x)
  \left(\bar\psi^{T\bar{a}}_{1,\bar\alpha}(C\gamma_5)_{\bar{\alpha}\bar{\beta}}\bar{\psi}^b_{2,\bar\beta}\right)
        \bar{\psi}^{\bar{c}}_{3,\gamma}(0)
  \rangle \\
 &= \varepsilon^{abc}\varepsilon^{\bar{a}\bar{b}\bar{c}} 
    (C\gamma_5)_{\alpha\beta} (C\gamma_5)_{\bar{\alpha}\bar{\beta}}
    G^{(1)}(x,0)^{a\bar{a}}_{\alpha\bar{\alpha}}
    G^{(2)}(x,0)^{b\bar{b}}_{\beta\bar{\beta}}
    G^{(3)}(x,0)^{c\bar{c}}_{\gamma\bar{\gamma}}
\end{split}
\end{gather*}

\begin{gather*}
\begin{split}
\langle B^{123}_\gamma(x) \bar{B}^{132}_{\bar\gamma}(0)\rangle
 &= -\varepsilon^{abc}\varepsilon^{\bar{a}\bar{b}\bar{c}} \langle
  \left(\psi^{T a}_{1,\alpha}(C\gamma_5)_{\alpha\beta}\psi^b_{2,\beta}\right)
        \psi^c_{3,\gamma}(t)
  \left(\bar\psi^{T\bar{a}}_{1,\bar\alpha}(C\gamma_5)_{\bar{\alpha}\bar{\beta}}\bar{\psi}^b_{3,\bar\beta}\right)
        \bar{\psi}^{\bar{c}}_{2,\gamma}(0)
  \rangle \\
 &=-\varepsilon^{abc}\varepsilon^{\bar{a}\bar{b}\bar{c}} 
    (C\gamma_5)_{\alpha\beta} (C\gamma_5)_{\bar{\alpha}\bar{\beta}}
    G^{(1)}(x,0)^{a\bar{a}}_{\alpha\bar{\alpha}}
    G^{(2)}(x,0)^{b\bar{c}}_{\beta\bar{\gamma}}
    G^{(3)}(x,0)^{c\bar{b}}_{\gamma\bar{\beta}} \\
 &= \varepsilon^{abc}\varepsilon^{\bar{a}\bar{b}\bar{c}} 
    (C\gamma_5)_{\alpha\beta} (C\gamma_5)_{\bar{\alpha}\bar{\beta}}
    G^{(1)}(x,0)^{a\bar{a}}_{\alpha\bar{\alpha}}
    G^{(2)}(x,0)^{b\bar{b}}_{\beta\bar{\gamma}}
    G^{(3)}(x,0)^{c\bar{c}}_{\gamma\bar{\beta}}
\end{split}
\end{gather*}
and similarly we will consider $C\gamma_5\rightarrow iC\gamma_k$ with labelling
$B_{k,\gamma}$ for the decuplet baryons.

With a suitable $T_{\alpha\beta}$ as some generic $4\times 4$ matrix in
Dirac spin space, and $\alpha,\beta$ are Dirac indices. We see that 
if we contract all the spin and color indices we arrive at the expressions
%
\begin{gather*}
\begin{split}
T_{\bar{\gamma}\gamma}& \langle B^{123}_\gamma(x) \bar{B}^{123}_{\bar\gamma}(0)\rangle\\
 &= {\rm tr}(T * {\rm tr}_C(G^{(3)}(x,0)
  * {\rm tr}_S({\rm quarkContract13}(G^{(1)}(x,0)(C\gamma_5),(C\gamma_5)G^{(2)}(x,0)))))
\end{split}
\end{gather*}
%
\begin{gather*}
\begin{split}
T_{\bar{\gamma}\gamma}& \langle B^{123}_\gamma(x) \bar{B}^{132}_{\bar\gamma}(0)\rangle\\
 &= {\rm tr}(T * {\rm tr}_C(G^{(3)}(x,0)
  * {\rm quarkContract13}(G^{(1)}(x,0)(C\gamma_5),(C\gamma_5)G^{(2)}(x,0))))
\end{split}
\end{gather*}
%
where ${\rm tr}_C$, ${\rm tr}_S$ and ${\rm tr}$ are traces in 
color, spin and all color-spin indices, resp.

\newpage

\section{Baryon 2-point correlation functions}

\subsection{$\Sigma^+\Sigma^+$, or $PP$:}
For the case of the $\Sigma^+$/proton: $\Sigma^+_\gamma = B^{suu}_\gamma$ 
(for proton: $s\rightarrow d$). Note that $B^{123}_\gamma = -B^{213}_\gamma$.
%
\begin{gather*}
\begin{split}
\langle \Sigma^+_\gamma(x) \bar{\Sigma}^+_{\bar\gamma}(0)\rangle
 &= \langle B^{123}_\gamma(x) \bar{B}^{123}_{\bar\gamma}(0)\rangle_{\stackrel{1=s}{2,3=u}}
  + \langle B^{123}_\gamma(x) \bar{B}^{132}_{\bar\gamma}(0)\rangle_{\stackrel{1=s}{2,3=u}}
\end{split}
\end{gather*}
The contractions needed here consist of the following:
%
\begin{gather*}
\begin{split}
T_{\bar{\gamma}\gamma} \langle B^{123}_\gamma(x) \bar{B}^{123}_{\bar{k},\bar{\gamma}}(0)\rangle\\
 &= \varepsilon^{abc}\varepsilon^{\bar{a}\bar{b}\bar{c}} 
    (C\gamma_5)_{\alpha\beta} (C\gamma_5)_{\bar{\alpha}\bar{\beta}}
    G^{(1)}(x,0)^{a\bar{a}}_{\alpha\bar{\alpha}}
    G^{(2)}(x,0)^{b\bar{b}}_{\beta\bar{\beta}}
    G^{(3)}(x,0)^{c\bar{c}}_{\gamma\bar{\gamma}}\\
 &= {\rm tr}(T * {\rm tr}_C(G^{(3)}(x,0)
  * {\rm tr}_S({\rm qC13}(G^{(1)}(x,0)(C\gamma_5),(C\gamma_5)G^{(2)}(x,0)))))
\end{split}
\end{gather*}

\begin{gather*}
\begin{split}
T_{\bar{\gamma}\gamma} \langle B^{123}_\gamma(x) \bar{B}^{132}_{\bar{k},\bar{\gamma}}(0)\rangle\\
 &= \varepsilon^{abc}\varepsilon^{\bar{a}\bar{b}\bar{c}} 
    (C\gamma_5)_{\alpha\beta} (C\gamma_5)_{\bar{\alpha}\bar{\beta}}
    G^{(1)}(x,0)^{a\bar{a}}_{\alpha\bar{\alpha}}
    G^{(2)}(x,0)^{b\bar{b}}_{\beta\bar{\gamma}}
    G^{(3)}(x,0)^{c\bar{c}}_{\gamma\bar{\beta}}\\
 &= {\rm tr}(T * {\rm tr}_C(G^{(3)}(x,0)
 * {\rm qC13}(G^{(1)}(x,0)(C\gamma_5),(C\gamma_5)G^{(2)}(x,0))))
\end{split}
\end{gather*}
%
Thus, the explicit final form for quarks $1,2,3$ being $d$, $u$ and $u$,
the correlation function is
%
\begin{gather*}
\begin{split}
T  \langle N(x) \bar{N}(0)\rangle
 &= {\rm tr}(T * {\rm tr}_C(U(x,0)
  * {\rm tr}_S({\rm qC13}(D(x,0)(C\gamma_5),(C\gamma_5)U(x,0)))))\\
 &+ {\rm tr}(T * {\rm tr}_C(U(x,0)
 * {\rm qC13}(D(x,0)(C\gamma_5),(C\gamma_5)U(x,0))))
\end{split}
\end{gather*}


\subsection{$\Sigma^{\ast +}\Sigma^{\ast +}$, or $\Delta^+\Delta^+$:}
Here $\Sigma^{\ast+}/\Delta^+$: 
$\Sigma^{\ast+}_{k,\gamma} = 2 B^{suu}_{k,\gamma} + B^{uus}_{k,\gamma}$ 
(for $\Delta^+$: $s\rightarrow d$). Note that $B^{213}_{k,\gamma}=B^{123}_{k,\gamma}$.
%
\begin{gather*}
\begin{split}
\langle \Sigma^{\ast+}_{k,\gamma}(x) \bar{\Sigma}^{\ast+}_{\bar{k},\bar{\gamma}}(0)\rangle
 &= 4\langle B^{123}_{k,\gamma}(x) \bar{B}^{123}_{\bar{k},\bar{\gamma}}(0)\rangle_{\stackrel{1=s}{2,3=u}}
  + 2\langle B^{123}_{k,\gamma}(x) \bar{B}^{123}_{\bar{k},\bar{\gamma}}(0)\rangle_{\stackrel{1,2=u}{3=s}}\\
 &+ 4\langle B^{123}_{k,\gamma}(x) \bar{B}^{132}_{\bar{k},\bar{\gamma}}(0)\rangle_{\stackrel{1=s}{2,3=u}}
  + 4\langle B^{123}_{k,\gamma}(x) \bar{B}^{132}_{\bar{k},\bar{\gamma}}(0)\rangle_{\stackrel{1,3=u}{2=s}}\\
 &+ 4\langle B^{123}_{k,\gamma}(x) \bar{B}^{132}_{\bar{k},\bar{\gamma}}(0)\rangle_{\stackrel{1,2=u}{3=s}}
\end{split}
\end{gather*}


\subsection{$\Delta^+\Delta^+$:}
This correlator is the same as the $\Sigma^{\ast+}\Sigma^{\ast+}$. In the
case the $s\rightarrow d$, we have $\Delta^+$: 
$\Delta^{+}_{k,\gamma} = 2 B^{duu}_{k,\gamma} + B^{uud}_{k,\gamma}$. 
Note that $B^{213}_{k,\gamma}=B^{123}_{k,\gamma}$. For degenerate $u$ and $d$ quarks,
we have (up to a factor of 6)
%
\begin{gather*}
\begin{split}
\langle \Delta^{+}_{k,\gamma}(x) \bar{\Delta}^{+}_{\bar{k},\bar{\gamma}}(0)\rangle
 &=  \langle B^{123}_{k,\gamma}(x) \bar{B}^{123}_{\bar{k},\bar{\gamma}}(0)\rangle
  + 2\langle B^{123}_{k,\gamma}(x) \bar{B}^{132}_{\bar{k},\bar{\gamma}}(0)\rangle
\end{split}
\end{gather*}


\subsection{$\Lambda\Lambda$:}
Here
$\Lambda_{\gamma} = 2 B^{uds}_{\gamma} + B^{sdu}_{\gamma} + B^{usd}_\gamma$ 
where $B^{213}_{\gamma}=-B^{123}_{\gamma}$.
%
\begin{gather*}
\begin{split}
\langle \Lambda_{\gamma}(x) \bar{\Lambda}_{\bar{\gamma}}(0)\rangle
 &= 4\langle B^{123}_{\gamma}(x) \bar{B}^{123}_{\bar{\gamma}}(0)\rangle_{1=u,2=d,3=s}\\
 &+  \langle B^{123}_{\gamma}(x) \bar{B}^{123}_{\bar{\gamma}}(0)\rangle_{1=s,2=d,3=u}
  +  \langle B^{123}_{\gamma}(x) \bar{B}^{123}_{\bar{\gamma}}(0)\rangle_{1=s,2=u,3=d}\\
 &+ 2\langle B^{123}_{\gamma}(x) \bar{B}^{132}_{\bar{\gamma}}(0)\rangle_{1=d,2=u,3=s}
  + 2\langle B^{123}_{\gamma}(x) \bar{B}^{132}_{\bar{\gamma}}(0)\rangle_{1=u,2=d,3=s}\\
 &+ 2\langle B^{123}_{\gamma}(x) \bar{B}^{132}_{\bar{\gamma}}(0)\rangle_{1=d,2=s,3=u}
  + 2\langle B^{123}_{\gamma}(x) \bar{B}^{132}_{\bar{\gamma}}(0)\rangle_{1=u,2=s,3=d}\\
 &-  \langle B^{123}_{\gamma}(x) \bar{B}^{132}_{\bar{\gamma}}(0)\rangle_{1=s,2=d,3=u}
  -  \langle B^{123}_{\gamma}(x) \bar{B}^{132}_{\bar{\gamma}}(0)\rangle_{1=s,2=u,3=d}
\end{split}
\end{gather*}

\subsection{$\Delta^+ P$:}
Here $P_\gamma = B^{duu}_\gamma$ where $B^{123}_\gamma = -B^{213}_\gamma$.
$\Delta^+_{k,\gamma} = 2 B^{duu}_{k,\gamma} + B^{uud}_{k,\gamma}$ 
where $B^{213}_{k,\gamma}=B^{123}_{k,\gamma}$.
%
\begin{gather*}
\begin{split}
\langle P_{\gamma}(x) \bar{\Delta}^+_{\bar{k},\bar{\gamma}}(0)\rangle
 &= 2\langle B^{d u_1 u_2}_{\gamma}(x) \bar{B}^{d u_1 u_2}_{\bar{k},\bar{\gamma}}(0)\rangle
  + 2\langle B^{d u_1 u_2}_{\gamma}(x) \bar{B}^{d u_2 u_1}_{\bar{k},\bar{\gamma}}(0)\rangle\\
 &+  \langle B^{d u_1 u_2}_{\gamma}(x) \bar{B}^{u_1 u_2 d}_{\bar{k},\bar{\gamma}}(0)\rangle
  +  \langle B^{d u_1 u_2}_{\gamma}(x) \bar{B}^{u_2 u_1 d}_{\bar{k},\bar{\gamma}}(0)\rangle\\
 &= 2\langle B^{123}_{\gamma}(x) \bar{B}^{123}_{\bar{k},\bar{\gamma}}(0)\rangle_{1=d,2=u_1,3=u_2}
  + 2\langle B^{123}_{\gamma}(x) \bar{B}^{132}_{\bar{k},\bar{\gamma}}(0)\rangle_{1=d,2=u_1,3=u_2}\\
 &-  \langle B^{123}_{\gamma}(x) \bar{B}^{132}_{\bar{k},\bar{\gamma}}(0)\rangle_{1=u_1,2=u_2,3=d}
  -  \langle B^{123}_{\gamma}(x) \bar{B}^{132}_{\bar{k},\bar{\gamma}}(0)\rangle_{1=u_1,2=d,3=u_2}
\end{split}
\end{gather*}
The contractions needed here consist of the following:
%
\begin{gather*}
\begin{split}
T_{\bar{\gamma}\gamma} \langle B^{123}_\gamma(x) \bar{B}^{123}_{\bar{k},\bar{\gamma}}(0)\rangle\\
 &= \varepsilon^{abc}\varepsilon^{\bar{a}\bar{b}\bar{c}} 
    (C\gamma_5)_{\alpha\beta} (C\gamma_k)_{\bar{\alpha}\bar{\beta}}
    G^{(1)}(x,0)^{a\bar{a}}_{\alpha\bar{\alpha}}
    G^{(2)}(x,0)^{b\bar{b}}_{\beta\bar{\beta}}
    G^{(3)}(x,0)^{c\bar{c}}_{\gamma\bar{\gamma}}\\
 &= {\rm tr}(T * {\rm tr}_C(G^{(3)}(x,0)
  * {\rm tr}_S({\rm qC13}(G^{(1)}(x,0)(C\gamma_k),(C\gamma_5)G^{(2)}(x,0)))))
\end{split}
\end{gather*}

\begin{gather*}
\begin{split}
T_{\bar{\gamma}\gamma} \langle B^{123}_\gamma(x) \bar{B}^{132}_{\bar{k},\bar{\gamma}}(0)\rangle\\
 &= \varepsilon^{abc}\varepsilon^{\bar{a}\bar{b}\bar{c}} 
    (C\gamma_5)_{\alpha\beta} (C\gamma_k)_{\bar{\alpha}\bar{\beta}}
    G^{(1)}(x,0)^{a\bar{a}}_{\alpha\bar{\alpha}}
    G^{(2)}(x,0)^{b\bar{b}}_{\beta\bar{\gamma}}
    G^{(3)}(x,0)^{c\bar{c}}_{\gamma\bar{\beta}}\\
 &= {\rm tr}(T * {\rm tr}_C(G^{(3)}(x,0)
 * {\rm qC13}(G^{(1)}(x,0)(C\gamma_k),(C\gamma_5)G^{(2)}(x,0))))
\end{split}
\end{gather*}


\subsection{$P \Delta^+$:}
Here $P_\gamma = B^{duu}_\gamma$ where $B^{123}_\gamma = -B^{213}_\gamma$.
$\Delta^+_{k,\gamma} = 2 B^{duu}_{k,\gamma} + B^{uud}_{k,\gamma}$ 
where $B^{213}_{k,\gamma}=B^{123}_{k,\gamma}$.
%
\begin{gather*}
\begin{split}
\langle \Delta^+_{k,\gamma}(x) \bar{P}_{\bar{\gamma}}(0)\rangle
 &= 2\langle B^{d u_1 u_2}_{k,\gamma}(x) \bar{B}^{d u_1 u_2}_{\bar{\gamma}}(0)\rangle
  + 2\langle B^{d u_1 u_2}_{k,\gamma}(x) \bar{B}^{d u_2 u_1}_{\bar{\gamma}}(0)\rangle\\
 &+  \langle B^{u_1 u_2 d}_{k,\gamma}(x) \bar{B}^{d u_1 u_2}_{\bar{\gamma}}(0)\rangle
  +  \langle B^{u_1 u_2 d}_{k,\gamma}(x) \bar{B}^{d u_2 u_1}_{\bar{\gamma}}(0)\rangle\\
 &= 2\langle B^{123}_{k,\gamma}(x) \bar{B}^{123}_{\bar{\gamma}}(0)\rangle_{1=d,2=u_1,3=u_2}
  + 2\langle B^{123}_{k,\gamma}(x) \bar{B}^{132}_{\bar{\gamma}}(0)\rangle_{1=d,2=u_1,3=u_2}\\
 &-  \langle B^{123}_{k,\gamma}(x) \bar{B}^{132}_{\bar{\gamma}}(0)\rangle_{1=u_1,2=u_2,3=d}
  -  \langle B^{123}_{k,\gamma}(x) \bar{B}^{132}_{\bar{\gamma}}(0)\rangle_{1=u_2,2=u_1,3=d}
\end{split}
\end{gather*}
The contractions needed here consist of the following:
%
\begin{gather*}
\begin{split}
T_{\bar{\gamma}\gamma} \langle B^{123}_{k,\gamma}(x) \bar{B}^{123}_{\bar{\gamma}}(0)\rangle\\
 &= \varepsilon^{abc}\varepsilon^{\bar{a}\bar{b}\bar{c}} 
    (C\gamma_k)_{\alpha\beta} (C\gamma_5)_{\bar{\alpha}\bar{\beta}}
    G^{(1)}(x,0)^{a\bar{a}}_{\alpha\bar{\alpha}}
    G^{(2)}(x,0)^{b\bar{b}}_{\beta\bar{\beta}}
    G^{(3)}(x,0)^{c\bar{c}}_{\gamma\bar{\gamma}}\\
 &= {\rm tr}(T * {\rm tr}_C(G^{(3)}(x,0)
  * {\rm tr}_S({\rm qC13}(G^{(1)}(x,0)(C\gamma_5),(C\gamma_k)G^{(2)}(x,0)))))
\end{split}
\end{gather*}

\begin{gather*}
\begin{split}
T_{\bar{\gamma}\gamma}\langle B^{123}_{k,\gamma}(x) \bar{B}^{132}_{\bar\gamma}(0)\rangle&\\
 &= \varepsilon^{abc}\varepsilon^{\bar{a}\bar{b}\bar{c}} 
    (C\gamma_k)_{\alpha\beta} (C\gamma_5)_{\bar{\alpha}\bar{\beta}}
    G^{(1)}(x,0)^{a\bar{a}}_{\alpha\bar{\alpha}}
    G^{(2)}(x,0)^{b\bar{b}}_{\beta\bar{\gamma}}
    G^{(3)}(x,0)^{c\bar{c}}_{\gamma\bar{\beta}}\\
 &= {\rm tr}(T * {\rm tr}_C(G^{(3)}(x,0)
  * {\rm qC13}(G^{(1)}(x,0)(C\gamma_5),(C\gamma_k)G^{(2)}(x,0))))
\end{split}
\end{gather*}


\newpage

\section{Baryon 3-point functions}

We now want to construct baryon 3-point functions with an electromagnetic
current
\[
J(x) = \sum_f e_f \bar{q}_f(x){\cal O}(x,y) q_f(y)
\]
where we sum over all flavors $f$, and $\cal O$ is composed of
some gamma matrices and some covariant displacement function -- 
possibly only a delta function. The 3-point functions take the form
%
\[
\Gamma^{(3)} = \langle B(x_2) J(x_1) \bar{B}(0) \rangle
\]
for some interpolating field $B(x)$. We see that the current is inserted on
each line in the 2-point function. 

We want expressions for the individual $u$, $d$, etc. quark contributions 
within the 3-point function. These can be formed directly from the 2-point
function using a generalized propagator of the form
\[
\widetilde{U}(x_2,x_1,0) = U(x_2,x_1) \Gamma U(x_1,0) 
   = \gamma_5 U^\dag(x_1,x_2)\gamma_5 \Gamma U(x_1,0)\ .
\]
for the illustrative case of the simple $u$-quark $\Gamma$-matrix
inserted current. The 3-point function is constructed from the 2-point
function after inserting into each line, resp. 

\subsection{Illustrative case $N J N$: $\bar{u}\Gamma u$ insertion}
Consider the case of the proton: $P_\gamma = B^{duu}_\gamma$ and neutron
$N_\gamma = B^{udd}_\gamma$. Note that $B^{123}_\gamma = -B^{213}_\gamma$.
The $u$ quark contribution to the proton 3-point function amounts to
summing $\widetilde{U}$ quark contributions in the 2-point function.
If one writes the 2-point correlator as a function of the quarks
%
\begin{gather*}
\begin{split}
\Gamma^{(2)}(D(x,0),&U_1(x,0),U_2(x,0)) = T  \langle P(x) \bar{P}(0)\rangle \\
 &= T \langle B^{123}(x) \bar{B}^{123}(0)\rangle_{\stackrel{1=d}{2,3=u}}
  + T \langle B^{123}(x) \bar{B}^{132}(0)\rangle_{\stackrel{1=d}{2,3=u}}\\
 &= {\rm tr}(T * {\rm tr}_C(U_2(x,0)
  * {\rm tr}_S({\rm qC13}(D(x,0)(C\gamma_5),(C\gamma_5)U_1(x,0)))))\\
 &+ {\rm tr}(T * {\rm tr}_C(U_2(x,0)
 * {\rm qC13}(D(x,0)(C\gamma_5),(C\gamma_5)U_1(x,0))))\quad,
\end{split}
\end{gather*}
then the $\bar{u}{\cal O}u$-insertion in the 3-point function is
\[
\Gamma^{(3)}_u(D,U_1,U_2) = \Gamma^{(2)}(D(x_2,0),\widetilde{U}(x_2,x_1,0),U_2(x_2,0))
 +  \Gamma^{(2)}(D(x_2,0),U_1(x_2,0),\widetilde{U}(x_2,x_1,0))
\]
where $\widetilde{U}(x_2,x_1,0) = U(x_2,x_1) {\cal O}(x_1,y) U(y,0)$.
Thus, in explicit form the $u$-quark contribution to the proton 3-point
function is
%
\begin{gather*}
\begin{split}
T  \langle P(x_2) &\ \bar{u}(x_1){\cal O}(x_1,y) u(y)\ \bar{P}(0)\rangle\\
 &= {\rm tr}(T * {\rm tr}_C(\widetilde{U}(x_2,x_1,0)
  * {\rm tr}_S({\rm qC13}(D(x_2,0)(C\gamma_5),(C\gamma_5)U(x_2,0)))))\\
 &+ {\rm tr}(T * {\rm tr}_C(\widetilde{U}(x_2,x_1,0)
 * {\rm qC13}(D(x_2,0)(C\gamma_5),(C\gamma_5)U(x_2,0)))) \\
 &+ {\rm tr}(T * {\rm tr}_C(U(x_2,0)
  * {\rm tr}_S({\rm qC13}(D(x_2,0)(C\gamma_5),(C\gamma_5)\widetilde{U}(x_2,x_1,0)))))\\
 &+ {\rm tr}(T * {\rm tr}_C(U(x_2,0)
 * {\rm qC13}(D(x_2,0)(C\gamma_5),(C\gamma_5)\widetilde{U}(x_2,x_1,0))))
\end{split}
\end{gather*}

The rest of the form factors follow in such a simple fashion. We tabulate them
for convenience.

\subsection{$p J p$}
Proton 2-point function is
\[
\Gamma^{(2)}(D,U,U) 
  = T \langle B^{123}(x) \bar{B}^{123}(0)\rangle_{\stackrel{1=d}{2,3=u}}
  + T \langle B^{123}(x) \bar{B}^{132}(0)\rangle_{\stackrel{1=d}{2,3=u}}
\]

\paragraph{$\bar{u}{\cal O}u$-insertion:}
\[
\Gamma^{(3)}_u(D,U,U) = \Gamma^{(2)}(D(x_2,0),\widetilde{U}(x_2,x_1,0),U(x_2,0))
 +  \Gamma^{(2)}(D(x_2,0),U(x_2,0),\widetilde{U}(x_2,x_1,0))
\]

\paragraph{$\bar{d}{\cal O}d$-insertion:}
\[
\Gamma^{(3)}_d(D,U,U) = \Gamma^{(2)}(\widetilde{D}(x_2,x_1,0),U(x_2,0),U(x_2,0))
\]

\subsection{$n J n$}
Neutron 2-point function is
\[
\Gamma^{(2)}(U,D,D) 
  = T \langle B^{123}(x) \bar{B}^{123}(0)\rangle_{\stackrel{1=u}{2,3=d}}
  + T \langle B^{123}(x) \bar{B}^{132}(0)\rangle_{\stackrel{1=u}{2,3=d}}
\]

\paragraph{$\bar{u}{\cal O}u$-insertion:}
\[
\Gamma^{(3)}_u(U,D,D) = \Gamma^{(2)}(\widetilde{U}(x_2,x_1,0),D(x_2,0),D(x_2,0))
\]

\paragraph{$\bar{d}{\cal O}d$-insertion:}
\[
\Gamma^{(3)}_d(U,D,D) = \Gamma^{(2)}(U(x_2,0),\widetilde{D}(x_2,x_1,0),D(x_2,0))
 +  \Gamma^{(2)}(U(x_2,0),D(x_2,0),\widetilde{D}(x_2,x_1,0))
\]


\subsection{$\Delta^+ P$:}
Here $P_\gamma = B^{duu}_\gamma$ where $B^{123}_\gamma = -B^{213}_\gamma$.
$\Delta^+_{k,\gamma} = 2 B^{duu}_{k,\gamma} + B^{uud}_{k,\gamma}$ 
where $B^{213}_{k,\gamma}=B^{123}_{k,\gamma}$. 
After contracting over the project $T$, the 2-point correlation function is
%
\begin{gather*}
\begin{split}
\Gamma^{(2)}(D,U,U) & = T \langle P(x) \bar{\Delta}^+_{\bar{k}}(0)\rangle\\
 &= 2\langle T B^{123}(x) \bar{B}^{123}_{\bar{k}}(0)\rangle_{\stackrel{1=d}{2,3=u}}
  + 2\langle T B^{123}(x) \bar{B}^{132}_{\bar{k}}(0)\rangle_{\stackrel{1=d}{2,3=u}}\\
 &-  \langle T B^{123}(x) \bar{B}^{132}_{\bar{k}}(0)\rangle_{\stackrel{1,2=u}{3=d}}
  -  \langle T B^{123}(x) \bar{B}^{132}_{\bar{k}}(0)\rangle_{\stackrel{1,3=u}{2=d}}
\end{split}
\end{gather*}

\paragraph{$\bar{u}{\cal O}u$-insertion:}
\[
\Gamma^{(3)}_u(D,U,U) = \Gamma^{(2)}(D(x_2,0),\widetilde{U}(x_2,x_1,0),U(x_2,0))
 +  \Gamma^{(2)}(D(x_2,0),U(x_2,0),\widetilde{U}(x_2,x_1,0))
\]

\paragraph{$\bar{d}{\cal O}d$-insertion:}
\[
\Gamma^{(3)}_d(D,U,U) = \Gamma^{(2)}(\widetilde{D}(x_2,x_1,0),U(x_2,0),U(x_2,0))
\]


\newpage

\section{Sequential Sources}

{\bf NEED SOME INTRO HERE ON THE SOURCE TRICK - FIDDLING WITH APPLYING
DIRAC OP AT $x_1$}

As a final step, \verb|Chroma| takes the sequential source from below and
does  $\gamma_5 S^\dagger \gamma_5$.

\subsection{$N J N$}
Proton 2-point function is
\[
\Gamma^{(2)}(D,U,U) 
  = T \langle B^{123}(x) \bar{B}^{123}(0)\rangle_{\stackrel{1=d}{2,3=u}}
  + T \langle B^{123}(x) \bar{B}^{132}(0)\rangle_{\stackrel{1=d}{2,3=u}}
\]

\paragraph{Sequential source for $u$-quark:}
\begin{gather*}
\begin{split}
X &= {\rm qC24}(U(C\gamma_5),(C\gamma_5)D) \\
S_u(x_2) &= T*X + {\rm tr}_S(X)*T
  - {\rm qC13}((C\gamma_5) U (C\gamma_5),U T)
  - {\rm transposeSpin}({\rm qC12}(U T, (C\gamma_5) U (C\gamma_5)))
\end{split}
\end{gather*}

\paragraph{Sequential source for $d$-quark:}
\[
S_d(x_2) = 
  - {\rm qC14}(T U(C\gamma_5),(C\gamma_5)U)
  - {\rm transposeSpin}({\rm qC12}(U T,(C\gamma_5)U (C\gamma_5)))
\]


\subsection{$\Delta^+ P$:}


\end{document}
